\appendix
\chapter{Appendix of Code Files}
\begin{table}[h]
\centering
    \begin{tabular}{@{}p{4cm} p{11cm} @{}} \toprule
  	File Name& Description\\ 
  	act.java&It stores activation functions.\\ 
  	dataIter.java&It reads in dependency trees and produces a data matrix in each batch iteration.\\ 
  	DataShuffler.java&It shuffles the original treebank.\\ 
  	EdgeFeaturizer\_NN.java&It transforms arcs into data representations.\\  
  	EdgeScorer\_NN.java&It scores every pair of arcs in a sentence given a neural network model.\\ 
  	Model\_NN.java&It stores information about word embeddings, postag embeddings, dictionary of dependency types, and dictionary of postags.\\ 
  	Parser\_NN.java&It is a replicate of Parser.java which is made compatible with the neural network model.\\ 
  	stats.java&It stores global methods.\\ 
  	Trainer\_NN.java&It trains the neural network.\\ 
  	VocReader.java&It reads the pretrained word embeddings into memory.\\ 
  	cleanData.py &It preprocesses the original data.\\
  	eval.py&It evaluates the performance of dependency parser excluding punctuations.\\
  	normalData.py& It standardizes the GloVe word embeddings.\\
  	\bottomrule 
    \end{tabular}
\caption{the Description of created Files}
\end{table}
\begin{table}[h]
\centering
    \begin{tabular}{@{}p{4cm} p{11cm} @{}} \toprule
  	File Name& Description\\ 
 	CoNLLReader.java& It reads in dependency trees from a text file.\\
 	CoNLLTree.java& It creates the data structure for a dependency tree.\\
 	CoNLLWriter.java& It writes predicted dependency trees into a text file.\\
 	Parser\_NN.java& It performs Eisner algorithm.\\
 	Table.java& It creates the data structure for the dictionary of dependency types and the dictionary of postags.\\
 	\bottomrule 
    \end{tabular}  
\caption{the Description of existed Files}
\end{table}

\chapter{Appendix of Dependency Labels}
Table \ref{tab:dt} provides description of dependency labels provided by Surdeanu et al. \cite{surdeanu2008conll}, Johansson and Richard \cite{johansson2008dependency}. Labels might be combined such as ADV-GSP.
\begin{table}[H]
\centering
    \begin{tabular}{@{}l p{12cm} @{}} \toprule
    Label&Description \\
ADV& General adverbial\\
AMOD& Modifier of adjective or adverbial\\
APPO& Apposition\\
BNF&  Benefactor complement (for) in dative shift\\
CONJ& Second conjunct (dependent on conjunction)\\
COORD& Coordination\\
DEP& Unclassified relation\\
DIR&  Adverbial of direction\\
DTV&  Dative complement (to) in dative shift\\
EXT&  Adverbial of extent\\
EXTR& Extraposed element in expletive constructions\\
GAP& Gapping: between conjunction and the parts of a structure with an ellipsed head\\
HMOD& Modifier in hyphenation, such as two in two-part\\
HYPH& Between first part of hyphenation and hyphen\\
IM& Infinitive verb (dependent on infinitive marker to)\\
LGS& Logical subject\\
LOC& Locative adverbial or nominal modifier\\
MNR& Adverbial of manner\\
NAME& Name-internal link\\
NMOD& Modifier of nominal\\
OBJ& Direct or indirect object or clause complement\\
OPRD& Object complement\\
P& Punctuation\\
PMOD& Between preposition and its child in a PP\\
POSTHON&  Posthonorific modifier of nominal such as Jr, Inc.\\
PRD& Predicative complement\\
PRN& Parenthetical\\
 	\bottomrule 
    \end{tabular}  
\end{table}
\begin{table}[H]
\centering
    \begin{tabular}{@{}l p{12cm} @{}} \toprule
    Label&Description \\
PRP&  Adverbial of purpose or reason\\
PRT& Particle\\
PUT& Various locative complements of the verb put\\    
ROOT& Root\\
SBJ& Subject\\
SUB& Subordinated clause (dependent on subordinating conjuction)\\
SUFFIX& Possessive suffix (dependent on possessor)\\
TITLE& Titles such as Mr, Dr\\
TMP& Temporal adverbial or nominal modifier\\
VC& Verb chain\\
VOC& Vocative\\
 	\bottomrule 
    \end{tabular}  
\caption{Description of Dependency Types}\label{tab:dt}
\end{table}